\documentclass{article}
\author{Chris Li}
\title{LBT-LUCI Documentation}

\begin{document}
\section{Before We Start}
Hey there. I hope this document will guide you through the rather complicated file system. I found it very difficult to write a complete guide to \emph{everything}, but at least this can be a guide to most of the things. If you ever need help, feel free to contact me via email.\\
\section{Summary of Work}
The research had three stages: Image reductions, Astrometry, and Photometry. I have arranged each stage into separated directories under bl3cq/iraf/.
\subsection{Image Reductions}
The first of three stages. We prepare the images out of the telescope such that they become usable for the next two stages. The images straight out of the telescope are not easy to work with \textbf{(compare)}. \\

1. For each object and for each filter, we take the raw unzipped 9 images through three automated scripts I wrote (trimflat, SEprepare,tweak) to prepare each of the nine images respectively. \\

2. To combine the nine images into one single image, we need to figure out the relative position between them. Begin by selecting an image (usually the first one) and manually select about 10+ stars in this template image, eyeball their centers in terms of x-y coordinate, record them in a file\textbf{(usually named)}.\\

3. We choose a particular star thats very well-behaved (with relatively small FWHM) in the template image that are also present in all other images, record its x-y coordinate in the template image, and record its x-y coordinates in other images. Write those coordinates down into a file and invoke \emph{calculate.py} to produce a file indicating relatively pixel shifts between them.\\

4. Feed the files produced in step 2 and 3 into \emph{immatch-imalign}, along with some other files you'd have to create to instruct imalign to output the correct file names, and imalign will align all nine images.\\

5. Use \emph{ccdred-combine} to combine the nine images into a single image. We call this image "reduced" for now.\\

6. \underline{Copy} this image to ~/iraf/reduced/. Enable write protection.
7. Delete unnecessary files generated during the process.\\

\subsection{Astrometry}
This is the second of three stages. We determine the best fit WCS coordinate on each oof the reduced imaages obtained from stage one. This process is very complicated and require lots of manual work. Thankfully most of the hard work is done except one object.
\subsection{Photometry}
This is the third of three stages. We try to establish the relation between instrumental magnitudes and reference true magnitudes, hoping to determine the unknown true magnitude of our object of interest. I didn't quite organize the process completely but I recommend you read\textbf{ the manual} and the hand-written document I gave you.
\section{General Structure}
The main file containing all the data is /iraf. Within it, several subdirectories and files:\\
\begin{enumerate}
\item login.cl: You know this.\\
\item login.cl.save: I believe this is a saved version of the default .cl.\\
\item mycl.cl: This file required for iraf to recognize the existence of customized scripts I wrote. See \textbf{section}\\
\item trashcan/: It's the trash can. Nothing important here.\\
\item uparm/: My parameter files. I hope you leave it unchanged because some of them will give you useful hints as to what to do precisely. For example, if you wonder what the format of a manually written file is, you could go lookup the parameter file of the task that uses it. Usually it will provide enough information so that you can find where it was last used and find the file of your interest, then copy its format.\\
\item cachedir/: leave alone. I have no idea what it does but apparently iraf needs it.\\
\item documents/: It has some interesting documents you might want to read.\\
\item rawdata/: What it says. Raw, unzipped data straight out of LBT. In their original names. I believe this directory is write-protected.\\
\item scripts\_n\_tasks/: This file contains all custom scripts I wrote for either iraf or Python.
\begin{enumerate}
\item buffer/: Buffers generated by emacs.\\
\item old/: Old and outdated scripts. They really should go to the trashcan.\\
\item poon/: The parameter files used by Poon. Used by SEprepare. \\
\item trimflat.cl: This is the first of the three automatic image reduction scripts. The second one is a terminal python script called SEprepare.py, and the third one is a cl task script called tweak.cl. All three scripts must (at least in my setup) reside in iraf/scripts\_n\_tasks/. This script will take raw images from the telescope and output the trimmed, flatfielded images with proper names and header keywords. It will also output a Flat image, and a "names.lis" file containing the names of all output images. The script calls \emph{flatcombine} and \emph{ccdproc} from \emph{imred-ccdproc}. Please load them before running this script.\\

Attention: this script was written for LBT luci images of the same object and the same filter types, and assumes a proper setinstrument setup(when in doubt,  ask Ricky). All raw images are assumed to NOT have an image type in the header. If there already exist an header type called "imtype", please delete it via \emph{epar hedit}.\\

Attention2: one of the outputs, names.lis, is a crucial file for subsequent scripts. Please do not delete this file until all three tasks are finished.\\

Produces: \begin{enumerate}
\item tag.W1234R.TF.XXXX.fits: tag(a desired tag. Not included in default setting), W(one-letter survey), 1234(RA), R(filter) TF(trimmed, flatfielded), XXXX(image number), fits(format).\\
\item names.lis: A file with the names of all output images. Required for subsequent tasks.\\
\item FlatR.fits: R(filter). The unnormalized flat field image. \\
\item logfile: A log file produced by ccdproc. Leave alone.
\end{enumerate}
\item SEprepare.py: This is the second of the three automatic image processing scripts. The first one is a cl task script called trimflat.cl, and the third one is a cl task script called tweak.cl.\\

All three scripts must reside in ~/iraf/scripts\_n\_tasks/.\\

Use: Just run python SEprepare.py in the terminal. Make sure you copied SEprepare.py from /iraf/scripts\_n\_tasks/ to where the images are and run the command there.\\

Input: trimmed, flatfielded images produced by trimflat, and name.lis. For each image, output its *Extracted Background*(XB) and *Subtracted Background with Objects Subtracted*(sXbo) images. These outputs will have proper names for the third script.\\

The script calls flatcombine and ccdproc from noao-imred-ccdproc. Please load them before running this script.\\

Attention: This script must be manually copied to where all the *TF*.fits are located in order to function correctly.\\

Attention2: This script relies on file "name.lis" produced by trimflat.cl.\\

Attention3: This step produces lots of files that will be used in the third stage. But after the third stage, they are useless. I highly recommend deleting them after stage 3.\\

Produces: \begin{enumerate}
\item poon.conv, poon.nnw, poon.param, poon.sex: Poon's parameter files for Sextractor. Can delete. They are copied directly from /iraf/scripts\_n\_tasks/poon/.\\
\item tag.W1234R.TFXB.XXXX.fits: Extracted Background image. tag(a desired tag. Not included in default setting), W(one-letter survey), 1234(RA), R(filter) TFXB(trimmed, flatfielded, Extracted Background), XXXX(image number), fits(format). \\
\item tag.W1234R.TFsXbo.XXXX.fits: Subtracted Background with Objects Subtracted images. tag(a desired tag. Not included in default setting), W(one-letter survey), 1234(RA), R(filter) TFsXbo(trimmed, flatfielded, Subtracted Background with Objects Subtracted), XXXX(image number), fits(format).\\
\item tag.W1234R.TF.XXXX.cat: The catalog file produced by Sextractor. I don't think it will be used.
\end{enumerate}
\item tweak.cl: Third script of the trio. The other two are trimflat.cl and SEprepare.py. Must be run last. This script smooths out the gradient in the background.\\

Documentation for this task is on the way...\\

Attention: It is known that sometimes hedit will fail to add "skyvalue" to the image header properly, but rather it will simply put "INDEF". Preliminary debugging yielded no result.
\item calculate.py: Documentation on the way...\\
\item dispemall.cl: A interesting script that I wrote to display a bunch of images all at once, starting at a desired window. Highly recommend you check it out!
\end{enumerate}
\item datared/: This is where the all the work happened. Very complicated:
\begin{enumerate}
\item image\_reduction/: where images were reduced. Inside you can find each object and each filters. \textbf{There is a subdirectory called "reduced/", containing all the reduced images.} The subdirectory reduced/ is write protected.\\
\item photometry/: photometric work here. Since I only did two of them and that you're likely going to redo all of them , I'll not document each file inside this directory (honestly because I don't remember what they are). You can refer back to this directory to see what I did if you are stuck.\\
\item astrometry/: astrometric work done here.
\begin{enumerate}
\item log.txt etc.: a detailed documentation of the status of images. Highly recommend a careful read.\\
\item wise0000+00/ etc.: each individual object and their astrometry files.\\
\item Gaia/: Contains all the Gaia catalogs. I carefully prepared them and they are all set. Don't change them. They are not write-protected yet.\\
\item other\_docs/: some of the remnants from older astrometric work. Leave them alone but I guess they are useless now.\\
\item astr\_reduced/: \textbf{Contains all the astrometrically reduced images. }Write protected.
\end{enumerate}
\end{enumerate}

\section{Process and How To}
\subsection{Image\_reduction/ Details}
Documentation on the Way
\subsection{Astrometry/ Details}
Documentation on the Way





\end{enumerate}








\end{document}