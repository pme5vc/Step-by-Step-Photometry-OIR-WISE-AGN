\documentclass[a4paper]{article}

\usepackage[a4paper,top=3cm,bottom=2cm,left=3cm,right=3cm,marginparwidth=1.75cm]{geometry}
\usepackage[english]{babel}
\usepackage[utf8x]{inputenc}
\usepackage[T1]{fontenc}

\title{Weekly Update \#3}
\author{Patrick Edwards}

\begin{document}
\date{Wednesday, May 31, 2017}
\maketitle

This week, I was not able to accomplish as much as I would have hoped.  I had to go home for the weekend for a doctor's appointment, and the internet at my house was not making it easy for me to do work through and ssh into the astronomy lab computers.  However I am currently working on completing some of the tasks I was assigned (at about 10am Wednesday as I am writing this). \\ \\
I first moved into what is hopefully the final iteration orm coor doing photometry on the wise0342+57 field.  Just using the fits images, I redid the process that we found was most effective.  Luckily, this seemed to only take about 20 minutes or so.  I have also read through the fitparams help page to learn more about how to operate this task, and ran it on the W0342 field. \\ \\
I also made a new directory for each of the astrometry-completed fields, and moved the corresponding fits images into each one, so I can begin working on the photometry for each of them.  The first field I started on was the W0404 field, but ran into the issue that there was only 6 2MASS sources in the field, and it looks like 2 of them are galaxies.  However I did go through and find the coordinates for each of the two images of the field, and worked with the 6 data points which were available.

\end{document}
